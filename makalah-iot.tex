\documentclass[
10pt, % Main document font size
a4paper, % Paper type, use 'letterpaper' for US Letter paper
oneside, % One page layout (no page indentation)
%twoside, % Two page layout (page indentation for binding and different headers)
headinclude,footinclude, % Extra spacing for the header and footer
BCOR5mm, % Binding correction
]{scrartcl}

%%%%%%%%%%%%%%%%%%%%%%%%%%%%%%%%%%%%%%%%%
% Arsclassica Article
% Structure Specification File
%
% This file has been downloaded from:
% http://www.LaTeXTemplates.com
%
% Original author:
% Lorenzo Pantieri (http://www.lorenzopantieri.net) with extensive modifications by:
% Vel (vel@latextemplates.com)
%
% License:
% CC BY-NC-SA 3.0 (http://creativecommons.org/licenses/by-nc-sa/3.0/)
%
%%%%%%%%%%%%%%%%%%%%%%%%%%%%%%%%%%%%%%%%%

%----------------------------------------------------------------------------------------
%	REQUIRED PACKAGES
%----------------------------------------------------------------------------------------

\usepackage[
nochapters, % Turn off chapters since this is an article        
beramono, % Use the Bera Mono font for monospaced text (\texttt)
eulermath,% Use the Euler font for mathematics
pdfspacing, % Makes use of pdftex’ letter spacing capabilities via the microtype package
dottedtoc % Dotted lines leading to the page numbers in the table of contents
]{classicthesis} % The layout is based on the Classic Thesis style

\usepackage{arsclassica} % Modifies the Classic Thesis package

\usepackage[T1]{fontenc} % Use 8-bit encoding that has 256 glyphs

\usepackage[utf8]{inputenc} % Required for including letters with accents

\usepackage{graphicx} % Required for including images
\graphicspath{{Figures/}} % Set the default folder for images

\usepackage{enumitem} % Required for manipulating the whitespace between and within lists

\usepackage{lipsum} % Used for inserting dummy 'Lorem ipsum' text into the template

\usepackage{subfig} % Required for creating figures with multiple parts (subfigures)

\usepackage{amsmath,amssymb,amsthm} % For including math equations, theorems, symbols, etc

\usepackage{varioref} % More descriptive referencing

%----------------------------------------------------------------------------------------
%	THEOREM STYLES
%---------------------------------------------------------------------------------------

\theoremstyle{definition} % Define theorem styles here based on the definition style (used for definitions and examples)
\newtheorem{definition}{Definition}

\theoremstyle{plain} % Define theorem styles here based on the plain style (used for theorems, lemmas, propositions)
\newtheorem{theorem}{Theorem}

\theoremstyle{remark} % Define theorem styles here based on the remark style (used for remarks and notes)

%----------------------------------------------------------------------------------------
%	HYPERLINKS
%---------------------------------------------------------------------------------------

\hypersetup{
%draft, % Uncomment to remove all links (useful for printing in black and white)
colorlinks=true, breaklinks=true, bookmarks=true,bookmarksnumbered,
urlcolor=webbrown, linkcolor=RoyalBlue, citecolor=webgreen, % Link colors
pdftitle={}, % PDF title
pdfauthor={\textcopyright}, % PDF Author
pdfsubject={}, % PDF Subject
pdfkeywords={}, % PDF Keywords
pdfcreator={pdfLaTeX}, % PDF Creator
pdfproducer={LaTeX with hyperref and ClassicThesis} % PDF producer
} % Include the structure.tex file which specified the document structure and layout

%% \hyphenation{Fortran hy-phen-ation} 

\title{\normalfont\spacedallcaps{Desain dan Implementasi Aplikasi IoT Untuk Sensor Cuaca}}

%\subtitle{Subtitle} 

\author{\spacedlowsmallcaps{Budi Rahardjo*}} 

%% \date{} 

\begin{document}

\renewcommand{\sectionmark}[1]{\markright{\spacedlowsmallcaps{#1}}} % The header for all pages (oneside) or for even pages (twoside)
%\renewcommand{\subsectionmark}[1]{\markright{\thesubsection~#1}} % Uncomment when using the twoside option - this modifies the header on odd pages
\lehead{\mbox{\llap{\small\thepage\kern1em\color{halfgray} \vline}\color{halfgray}\hspace{0.5em}\rightmark\hfil}} % The header style

\pagestyle{scrheadings} % Enable the headers specified in this block

%----------------------------------------------------------------------------------------
%	TABLE OF CONTENTS & LISTS OF FIGURES AND TABLES
%----------------------------------------------------------------------------------------

\maketitle % Print the title/author/date block

\setcounter{tocdepth}{2} % Set the depth of the table of contents to show sections and subsections only

\tableofcontents % Print the table of contents
\listoffigures % Print the list of figures
\listoftables % Print the list of tables

%	ABSTRACT

\section*{Abstrak} 
Makalah ini menjabarkan desain dan implementasi IoT untuk aplikasi sensor cuaca.

%	AUTHOR AFFILIATIONS

\let\thefootnote\relax\footnotetext{* \textit{Institut Teknologi Bandung}}


\newpage % Start the article content on the second page, 


\section{Pendahuluan}
Salah satu aplikasi IoT ({\em Internet of Things}) yang paling mudah
diimplementasikan adalah sensor cuaca. Sudah ada banyak tutorial tersedia untuk
aplikasi seperti ini, tetapi pembahasan yang lebih rinci mengenai desain dan
implementasinya belum tersedia.

\section{Boards}
Salah satu komponen utama dari IoT adalah {\em development boards} yang
digunakan sebagai basis. {\em Board} yang paling banyak digunakan adalah
Arduino\footnote{Informasi mengenai Arduino tersedia di situs arduino.cc.}.
Kesuksesan dari Arduino adalah keterbukaan desain dari {\em board} (hardware)
dan software (dalam bentuk Arduino IDE).

{\em Board} yang kemudian terkenal adalah board yang berbasis ESP8266.
Kelebihan dari board berbasis ESP8266 adalah board tersebut sudah memiliki
modul WiFi. Salah satu implementasi board ESP8266 yang terkenal adalah NodeMCU.
Ada beberapa versi dari NodeMCU yang tersedia, misal dari Amica dan Wemos.
Selain boards tersebut ada board buatan Indonesia, ESPectro, yang dikembangkan
oleh DycodeX. Pada implementasi ini kami menggunakan NodeMCU dikarenakan
ketersediaan komponen dan harganya yang relatif murah. Sebagai contoh, pada
saat sistem dikembangkan NodeMCU Wemos D1 mini memiliki harga Rp.~55.000,-.

[Foto NodeMCU]

\section{Sensor}
\subsection{DHT11}

\section{Aplikasi}

\paragraph{Paragraph Description} \lipsum[7] % Dummy text

\paragraph{Different Paragraph Description} \lipsum[8] % Dummy text

%------------------------------------------------

\section{Penutup}


%	BIBLIOGRAPHY

\renewcommand{\refname}{\spacedlowsmallcaps{References}} % For modifying the bibliography heading

\bibliographystyle{unsrt}

\bibliography{referensi.bib} % The file containing the bibliography


\end{document}
