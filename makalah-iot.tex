\documentclass[
10pt, % Main document font size
a4paper, % Paper type, use 'letterpaper' for US Letter paper
oneside, % One page layout (no page indentation)
%twoside, % Two page layout (page indentation for binding and different headers)
headinclude,footinclude, % Extra spacing for the header and footer
BCOR5mm, % Binding correction
]{scrartcl}

\input{structure.tex} % Include the structure.tex file which specified the document structure and layout

%% \hyphenation{Fortran hy-phen-ation} 

\title{\normalfont\spacedallcaps{Desain dan Implementasi Aplikasi IoT Untuk Sensor Cuaca}}

%\subtitle{Subtitle} 

\author{\spacedlowsmallcaps{Budi Rahardjo*}} 

%% \date{} 

\begin{document}

\renewcommand{\sectionmark}[1]{\markright{\spacedlowsmallcaps{#1}}} % The header for all pages (oneside) or for even pages (twoside)
%\renewcommand{\subsectionmark}[1]{\markright{\thesubsection~#1}} % Uncomment when using the twoside option - this modifies the header on odd pages
\lehead{\mbox{\llap{\small\thepage\kern1em\color{halfgray} \vline}\color{halfgray}\hspace{0.5em}\rightmark\hfil}} % The header style

\pagestyle{scrheadings} % Enable the headers specified in this block

%----------------------------------------------------------------------------------------
%	TABLE OF CONTENTS & LISTS OF FIGURES AND TABLES
%----------------------------------------------------------------------------------------

\maketitle % Print the title/author/date block

\setcounter{tocdepth}{2} % Set the depth of the table of contents to show sections and subsections only

\tableofcontents % Print the table of contents
\listoffigures % Print the list of figures
\listoftables % Print the list of tables

%	ABSTRACT

\section*{Abstrak} 
Makalah ini menjabarkan desain dan implementasi IoT untuk aplikasi sensor cuaca.

%	AUTHOR AFFILIATIONS

\let\thefootnote\relax\footnotetext{* \textit{Institut Teknologi Bandung}}


\newpage % Start the article content on the second page, 


\section{Pendahuluan}
Salah satu aplikasi IoT ({\em Internet of Things}) yang paling mudah
diimplementasikan adalah sensor cuaca. Sudah ada banyak tutorial tersedia untuk
aplikasi seperti ini, tetapi pembahasan yang lebih rinci mengenai alasan-alasan
dalam desain (design decisions) dan implementasinya belum tersedia. Kebanyakan
tutorial yang ada hanya fokus kepada satu hal saja dan hanya pada tahap awal.

Aplikasi yang dibahas pada makalah ini adalah sensor cuaca ({\em weather
sensor}) yang akan digunakan di kota Bandung. Sistem ini nantinya dapat
dikembangkan lebih lanjut untuk menjadi sensor lingkungan ({\em environment
sensors}). Sistem ini dapat digunakan menjadi bagian dari sebuah {\em smart
city}.

\section{Boards}
Salah satu komponen utama dari IoT adalah {\em development boards} yang
digunakan sebagai basis. {\em Board} yang paling banyak digunakan adalah
Arduino\footnote{Informasi mengenai Arduino tersedia di situs arduino.cc.}.
Kesuksesan dari Arduino adalah keterbukaan desain dari {\em board} (hardware)
dan software (dalam bentuk Arduino IDE). Akibatnya banyak orang dapat
memproduksi board Arduino dengan harga yang terjangkau.

{\em Board} yang kemudian terkenal adalah board yang berbasis ESP8266.
Kelebihan dari board berbasis ESP8266 adalah board tersebut sudah memiliki
modul WiFi (802.11 b/g/n, dengan WPA/WPA2), yang akan memudahkan dalam
menghubungkan perangkat ini ke jaringan.  (Pembahasan mengenai aspek jaringan
yang lebih rinci ada pada bagian terpisah.)

Ada banyak implementasi dari board berbasis ESP8266 ini, antara lain dari
Espressif, NodeMCU, ESPectro (dari DycodeX, sebuah perusahaan di Bandung), dan
masih banyak lainnya. Kepopuleran dari board berbasis ESP8266 ini membuat
harganya menjadi murah. Sebagai contoh, board WeMos D1 mini yang digunakan
dalam sistem ini memiliki harga Rp.~55.000,-.

Aspek harga sangat menentukan dalam pengembangan sebuah sistem. Untuk sistem
dengan jumlah titik (node) yang sedikit, perbedaan harga tidaklah terlalu
signifikan. Untuk sistem dengan jumlah titik (node) yang banyak, maka aspek
harga sangat menentukan. Sebagai contoh untuk sistem dengan 1000
titik~\footnote{Salah satu aplikasi yang terbayang adalah pemantauan dan
pengendalian lampu penerangan jalan yang jumlahnya ratusan ribu.}, dimana
masing-masing titik membutuhkan setidaknya satu board, perbedaan harga
Rp.~10.000,- akan menjadikan beda sepuluh juta rupiah. Padahal ada board yang
lebih bagus tetapi harganya berbeda cukup signifikan. Sebagai contoh, ada salah
satu board yang berharga Rp.~300.000,- untuk satu unitnya. Harapannya tentu
saja adalah harga ini akan turun sejalan dengan meningkatnya kemampuan
teknologi dan jumlah pengguna.


[Foto NodeMCU]

\section{Sensor}
\subsection{DHT11}
\subsection{DHT22}

\section{Jaringan}

\section{Aplikasi}
\subsection{Sisi Sensor}
\subsection{Sisi Server}

\section{Penutup}


%	BIBLIOGRAPHY

\renewcommand{\refname}{\spacedlowsmallcaps{References}} % For modifying the bibliography heading

\bibliographystyle{unsrt}

\bibliography{referensi.bib} % The file containing the bibliography


\end{document}
